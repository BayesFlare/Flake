\documentclass[letterpaper, 11pt]{article}
\usepackage{graphicx}
\usepackage{natbib}
\usepackage[left=3cm,top=3cm,right=3cm]{geometry}

\renewcommand{\topfraction}{0.85}
\renewcommand{\textfraction}{0.1}
\parindent=0cm

\title{RJObject Paper}
\author{Brendon J. Brewer}

\begin{document}
\maketitle
\abstract{This is the abstract.}

\section{Introduction}
Many inference problems have the following structure. There is some unknown
number, $N$, of objects in a region. Each of the $N$ objects has a property
$x$, which may be a single scalar value (for example, a mass), or a set of
values (e.g. a position and a mass).



\section{Sinusoidal Example}
Phase transition


\section{``Transit'' Example}




\end{document}

