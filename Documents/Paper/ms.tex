\documentclass[letterpaper, 11pt]{article}
\usepackage{graphicx}
\usepackage{natbib}
\usepackage[left=3cm,top=3cm,right=3cm]{geometry}

\renewcommand{\topfraction}{0.85}
\renewcommand{\textfraction}{0.1}
\parindent=0cm

\title{RJObject Paper}
\author{Brendon J. Brewer}

\begin{document}
\maketitle
\abstract{This is the abstract.}

\section{Introduction}
Many inference problems have the following structure. There is some unknown
number, $N$, of objects in a region. Each of the $N$ objects has a property
$x$, which may be a single scalar value (for example, a mass), or a set of
values (e.g. a position and a mass).

We would like to assign a prior to the objects' properties $\{x_i\}_{i=1}^N$.
Since $N$ may be large, it is usually easier to assign an ``interim prior''
conditional on some hyperparameters $\alpha$, and then assign a prior to
$\alpha$. This kind of model is usually called {\it hierarchical}.

The prior for $N$, $\alpha$, and $\{x_i\}$ is usually factorised
in the following way:

\begin{eqnarray}
p(N, \alpha, \{x_i\}) &=& p(N) p(\alpha | N) p(\{x_i\} | \alpha, N) \\
&=& p(N) p(\alpha) \prod_{i=1}^N p(\{x_i\} | \alpha).
\end{eqnarray}

Here we have assumed the priors for $N$ and $\alpha$ are independent, and
the interim prior for $\{x_i\}$ is iid and does not depend on $N$.


In other studies, it has been common to do separate runs of Nested Sampling
with different values of $N$. Then, the posterior for $N$ can be calculated
based on the estimates of the evidence or marginal likelihood. Our motivation
for using Nested Sampling is different. With reversible jump MCMC it is possible
to obtain the posterior for $N$ with a single run. However, the exploration
of the posterior may be very difficult, and sometimes the problem may even
contain a phase transition. We use Nested Sampling to overcome these
difficulties.


\section{Sinusoidal Example}
Phase transition


\section{``Transit'' Example}




\section*{Acknowledgements}
This work is supported by a Marsden Fast-Start grant
from the Royal Society of New Zealand. I would like to thank the following
people for valuable conversations and inspiration:
Anna Pancoast (UCSB), David Hogg (NYU), Daniel Foreman-Mackey (NYU),
Courtney Donovan (Auckland), Tom Loredo (),
John Skilling (MaxEnt Data Consultants), and Daniela Huppenkothen ().


\begin{thebibliography}{}
\bibitem[\protect\citeauthoryear{Brewer, P{\'a}rtay,
\& Cs{\'a}nyi}{2011b}]{dnest} Brewer B.~J., P{\'a}rtay L.~B., Cs{\'a}nyi G., 2011,
Statistics and Computing, 21, 4, 649-656. arXiv:0912.2380

\end{thebibliography}

\end{document}

